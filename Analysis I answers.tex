\documentclass[12pt]{article}
\usepackage{amsmath}
\usepackage{mathtools}
\newcommand{\increment}{\! + \! +}
\begin{document}
	\part*{Chapter 1}
	
	\section*{Section 1.1}
	
	N/A
	
	\section*{Section 1.2}
	
	N/A
	
	\part*{Chapter 2}
	
	\section*{Section 2.1}
	
	N/A
	
	\section*{Section 2.2}
	
	\subsection*{Exercise 2.2.1}
	
	We shall use induction on $a$, fixing $b$ and $c$. First, we shall prove the base case. If $a = 0$, $a + b = 0 + b = b$ by the definition of addition of natural numbers  (Definition 2.2.1). Therefore, $(a + b) + c = b + c$. Also by Definition 2.2.1, $a + (b + c) = 0 + (b + c) = b + c$. We now have $(a + b) + c = b + c = a + (b + c)$.
	
	Now suppose that $(a + b) + c = a + (b + c)$. We then have, by Definition 2.2.1, that $((a\increment) + b) + c = ((a + b)\increment) + c = ((a + b) + c)\increment = (a + (b + c))\increment$. Because of Definition 2.2.1 again, $(a + (b + c))\increment = (a\increment) + (b + c)$. Now we have $((a\increment) + b) + c = (a\increment) + (b + c)$, which closes the induction.
	
	\subsection*{Exercise 2.2.2}
	
	We shall use induction on $a$. The base case is vacuously true since $0$ is not positive. For the inductive step, we have to show that $a \increment$ being positive implies that $a\increment = b\increment$ for exactly one natural number $b$ if $a$ being positive implies that $a = c\increment$ for exactly one natural number $c$. But $a$ is equal to $b$ if $a \increment$ is equal to $b \increment$, by Axiom 2.4. This closes the induction.
	
	\subsection*{Exercise 2.2.3}
	
	(a) By Lemma 2.2.2, $a = a + 0$, so $a \ge a$. \\
	(b) Because $a \ge b$ and $b \ge c$, $a = b + n$ and $b = c + m$ for some $n$ and $m$. Therefore, $a = b + n = (c + m) + n$. By Proposition 2.2.5, $(c + m) + n = c + (m + n)$, so $a = c + (m + n)$. $n + m$ is a natural number, so $a \ge c$. \\
	(c) Because $a \ge b$ and $b \ge a$, $a = b + n$ and $b = a + m$. Therefore, $a = b + n = (a + m) + n$. By Proposition 2.2.5, $(a + m) + n = a + (m + n)$. Now we have $a = a + (m + n)$. By Lemma 2.2.2, $a = a + 0$, so we can use Proposition 2.2.6 to get $0 = m + n$. Using Corollary 2.2.9, we can deduce that $n$ and $m$ are both equal to $0$. Therefore, $a = b$. \\
	(d) If $a \ge b$, $a = b + d$ for some $d$. Because of this, $a + c = (b + d) + c$. By Propositions 2.2.4 and 2.2.5, $(b + d) + c = b + (d + c) = b + (c + d) = (b + c) + d$. Therefore, $a + c \ge b + c$.
	
	Conversely, if $a + c \ge b + c$, $a + c = (b + c) + d$ for some $d$. By Propositions 2.2.4 and 2.2.5, $(b + c) + d = b + (c + d) + b + (d + c) = (b + d) + c$. Using Proposition 2.2.6, we get $a = b + d$, and therefore, $a \ge b$. \\
	(f) If $a < b$, $b = a + c$ for some $c$, and $b \ne a$. Therefore, $c$ is positive, because otherwise $b = a + 0 = a$. ($a + 0 = a$ because of Lemma 2.2.2.)
	
	Conversely, if $b = a + c$ for some positive number $c$, we know that $a \le b$. If $a = b$, $a = a + c$. By Lemma 2.2.2, $a = a + 0$, so $a + 0 = a + c$. We can use Proposition 2.2.6 to get $0 = c$, but $c$ is positive. Therefore, $a \ne b$, and combined with $a \le b$, we get $a < b$. \\
	(e) We first prove that $a < b$ implies that $a \increment \le b$. By (f), $b = a + c$ for some positive $c$. By Lemma 2.2.10, exactly one $d$ exists such that $d \increment = c$. Therefore, $b = a + d \increment$. By Lemma 2.2.3 and Definition 2.2.1, $b = a + d \increment = (a + d) \increment = (a \increment) + d$. Therefore, $a \increment \le b$.
	
	Conversely, if $a \increment \le b$, we need to prove that $a < b$. Since $a \increment = a + 1 \ge a$, by (b), $a \le b$. If $a = b$, then $a \increment \le a$, and we get, from (c), that $a \increment = a$. This is a contradiction, from Axiom 2.3 if $a = 0$ and Axiom 2.4 otherwise. Therefore, $a < b$.
	
	\subsection*{Exercise 2.2.6 (unfinished)}
	
	We use induction on $n$. When $n = 0$, $m \le n$ implies that $0 = m + a$. But by Corollary 2.2.9, $m = 0 = n$, so $P(n)$ being true implies that $P(m)$ is true.
	
	If $a \le n$ and $P(n)$ together imply $P(a)$, we now need to prove that $b \le n \increment$ and $P(n \increment)$ together imply $P(b)$. Since $b \le n \increment$, $n \increment = b + c$ for some $c$. If $c$ is positive, by Lemma 2.2.10, $n \increment = b + d \increment$ for some $d$, so by Lemma 2.2.3 and Axiom 2.4, $n = b + d$, and $b \le n$. Therefore, $b \le n \increment$ implies that $b \le n$ or $n \increment = b + 0 = b$. Because of Lemmas 2.2.2 and 2.2.3, $n \increment = (n + 0) \increment = n + 0 \increment = n + 1$. Since $P(n \increment)$ implies $P(n)$, $P(n)$ is true. But then \textbf{?}
	
	\section*{Section 2.3}
	
	\subsection*{Exercise 2.3.1}
	
	First, we prove that $n \cdot 0 = 0$. We induct on $n$. When $n = 0$, $n \cdot 0 = 0 \cdot 0 = 0$, because of Definition 2.3.1. For the inductive hypothesis, we assume $n \cdot 0 = 0$. Then we need to prove that $n \increment \cdot 0 = 0$. By Definition 2.3.1, $n \increment \cdot 0 = n \cdot 0 + 0 = n \cdot 0$. But we already know that $n \cdot 0 = 0$, so $n \increment \cdot 0 = 0$. This closes the induction.
	
	Next, we prove that $n \cdot m \increment = nm + n$ using induction on $n$. When $n = 0$, $0 \cdot m \increment = 0 = 0 \cdot m + 0$. Now, if $n \cdot m \increment = nm + n$, we need to prove that $n \increment \cdot m \increment = n \increment \cdot m + n \increment$. We can deduce from $n \cdot m \increment = nm + n$ that $n \increment \cdot m \increment = (n \cdot m \increment) + m \increment = nm + n + m \increment = nm + (n + m) \increment = nm + n \increment \, + m = (n \increment) \cdot m + n \increment$. This closes the induction.
	
	Now we can prove that $nm = mn$. We induct on $m$. When $m = 0$, we have $n \cdot 0 = 0 = 0 \cdot n$. For the inductive hypothesis, we assume $nm = mn$. Now we need to prove $n \cdot m \increment = m \increment \cdot n$. We know that $n \cdot m \increment = nm + n$ and $m \increment \cdot n = mn + n$. But $nm = mn$, so $n \cdot m \increment = m \increment \cdot n$. This closes the induction.
	\subsection*{Exercise 2.3.4}
	By Proposition 3.4, $(a + b)^2 = (a + b)  \cdot (a + b) = (a + b) \cdot a + (a + b) \cdot b = a \cdot a + b \cdot a + a \cdot b + b \cdot b$. We can rewrite $b \cdot a + a \cdot b$ as $a \cdot b + a \cdot b$ because of Lemma 2.3.2. We can further rewrite this as $2ab$ because $2(ab) = 1(ab) + ab = 0(ab) + ab + ab = 0 + ab + ab = ab + ab$. We can also rewrite $a \cdot a$ and $b \cdot b$ as $a^2$ and $b^2$ respectively. Therefore, $(a + b)^2 = a^2 + 2ab + b^2$.
	\part*{Chapter 3}
	
	\section*{Section 3.1}
	
	\subsection*{Exercise 3.3.1}
	
	We know that at least one of the statements ``$a = c$'' and ``$a = d$'' is true since $\{a, b\} = \{c, d\}$. Similarly, at least one of the statements ``$b = c$'' and ``$b = d$'' is true. If we have that $a = c$ and $b = d$ at the same time, then we have proved what we want. (Same for $a = d$ and $b = c$ at the same time.) Otherwise, then we must have that $a = c = b$ or $a = d = b$. If both of these are true, then $a = b = c = d$, and therefore both $a = c$ and $b = d$. If only one of these is true, which we will assume to be the statement that $a = c = b$ as a similar argument holds assuming $a = d = b$, then $d \notin \{a, b\} = \{a, a\}$, while $d \in \{c, d\} = \{a, d\}$, so $\{a, b\} \ne \{c, d\}$, a contradiction. Therefore, if $\{a, b\} = \{c, d\}$, either both $a = c$ and $b = d$ or $a = d$ and $b = c$.
	
	\subsection*{Exercise 3.3.2}
	
	First, we show that these four sets exist. Axiom 3.3 tells us that $\emptyset$ exists. The singleton set axiom (part of Axiom 3.4) and Axiom 3.1 (so that we are allowed to construct the singleton set whose element is another set) tell us that $\{\emptyset\}$ exists. Applying them again gives that $\{\{\emptyset\}\}$ exists. Using the pair set axiom (also part of Axiom 3.4) and Axiom 3.1 allows us to create the set $\{\emptyset, \{\emptyset\}\}$.
	
	Now we show that these four sets are distinct. Since the sets $\{\emptyset\}$, $\{\{\emptyset\}\}$, and $\{\emptyset, \{\emptyset\}\}$ obviously all contain at least one element, if one of them is equal to $\emptyset$, then by Axiom 3.2, it contains no elements. But this contradicts that it has at least one element, and we have a contradiction. Now we need to prove that $\{\emptyset\}$ is not equal to $\{\{\emptyset\}\}$ or $\{\emptyset, \{\emptyset\}\}$. Both $\{\emptyset\} \in \{\{\emptyset\}\}$ and $\{\emptyset, \{\emptyset\}\}$ are true, but $\{\emptyset\} \notin \{\emptyset\}$. Therefore, by Axiom 3.2, $\{\emptyset\}$ is not equal to any of the other sets. (We have already proved that $\emptyset \ne \{\emptyset\}$, so $\{\emptyset\} \ne \emptyset$). The last thing that we need to prove now is that $\{\{\emptyset\}\} \ne \{\emptyset, \{\emptyset\}\}$. Since $\emptyset \in \{\emptyset, \{\emptyset\}\}$ and $\emptyset \notin \{\{\emptyset\}\}$, by Axiom 3.2, these two sets are not equal. Therefore, all four sets are distinct.
	
	\subsection*{Exercise 3.1.3}
	
	\subsubsection*{Proving that $\{a, b\} = \{a\} \cup \{b\}$}
	
	We shall prove that $\{a, b\} = \{a\} \cup \{b\}$. The following are equivalent:
	\begin{itemize}
		\item $c \in \{a, b\}$
		\item $c$ is either equal to $a$ or $b$
		\item $c \in \{a\}$ or $c \in \{b\}$
		\item $c \in \{a\} \cup \{b\}$
	\end{itemize}
	Therefore, $\{a, b\} = \{a\} \cup \{b\}$.
	
	\subsubsection*{Proving that $A \cup B = B \cup A$}
	
	We now shall prove that $A \cup B = B \cup A$. If $c \in A \cup B$, then at least one of the statements $c \in A$ and $c \in B$ are true. By Axiom 3.5, this means that $c \in B \cup A$. We can replace $A$ with $B$ and $B$ with $A$ to show the other direction. Therefore, $A \cup B = B \cup A$.
	
	\subsubsection*{Proving that $A \cup A = A \cup \emptyset = \emptyset \cup A = A$}
	
	Because it is impossible for $c$ to be in $\emptyset$, the following are equivalent:
	\begin{itemize}
		\item $c \in A$ or $c \in A$
		\item $c \in A$ or $c \in \emptyset$
		\item $c \in \emptyset$ or $c \in A$
		\item $c \in A$
	\end{itemize}
	These four statements are equivalent to $c \in A \cup A$, $c \in A \cup \emptyset$, $c \in \emptyset cup A$, and $c \in A$, respectively. Therefore, $A \cup A = A \cup \emptyset = \emptyset \cup A = A$.
	
	\subsection*{Exercise 3.1.11}
	
	The next paragraph will be written like a proof of Axiom 3.6 from Axiom 3.7: we will not define again $A$, $P(x)$, and so on.
	
	If $Q(x, y)$ is the statement that $x = y$ and $P(x)$ are both true, then by Axiom 3.7 and the fact that for each $x$ there exists at most one $y$ such that $Q(x, y)$, there exists a set $B = \{y: x = y \text{ and } P(x) \text{ for some } x \in A\}$. For any $z \in B$, we know that $z \in A$ and that $P(z)$ is true. For any $x \in A$ such that $P(x)$ is true, we similarly know that $z \in B$. Therefore, $A = B$.
	
	\section*{Section 3.2}
	
	\subsection*{Exercise 3.2.1}
	
	\subsubsection*{Proving that Axiom 3.9 implies Axiom 3.3}
	
	We can create a property $P(x)$ that is always false regardless to the choice of $x$. Then the set $\{x: P(x) \text{ is true}\}$ has no elements, as $P(x)$ is false all the time.
	
	\subsubsection*{Proving that Axiom 3.9 implies Axiom 3.4}
	
	If $P(x)$ is the property that $x = a$, then $\{x: P(x) \text{ is true}\}$ has only the element $a$. Similarly, if $Q(x)$ is the property that $x = a$ or $x = b$, then $\{x: Q(x) \text{ is true}\}$ has only the elements $a$ and $b$.
	
	\subsubsection*{Proving that Axiom 3.9 implies Axiom 3.5}
	
	If $P(x)$ is the property that $x \in A$ or $x \in B$, then $\{x: P(x) \text{ is true}\}$ exists, so $A \cup B$ exists.
	
	\subsubsection*{Proving that Axiom 3.9 implies Axiom 3.6}
	
	If $Q(x)$ is the property that $P(x)$ is true and $x \in A$, then $\{x: Q(x) \text{ is true}\} = \{x \in A: P(x)\}$ exists.
	
	\subsubsection*{Proving that Axiom 3.9 implies Axiom 3.7}
	
	If $Q(y)$ is the property that there exists some $x$ such that $P(x, y)$ is true, then $\{y: Q(y) \text{ is true}\} = \{y: P(x, y) \text{ is true for some x } \in A\}$ exists.
	
	\subsubsection*{Proving that Axiom 3.9 implies Axiom 3.8, assuming that all natural numbers are objects}
	
	If $P(x)$ is the property that $x$ is a natural number, then $\mathbf{N} = \{x: P(x) \text{ is true}\}$ exists, by Axiom 3.9.
	
	\section*{Section 3.3}
	
	\subsection*{Exercise 3.3.2}
	
	\subsubsection*{Proving that $f$ and $g$ being injective implies that $g \circ f$ is too}
	
	We will show that if $x \ne x'$, then $(g \circ f)(x) \ne (g \circ f)(x')$. First, since $f$ is injective, $f(x) \ne f(x')$. Since $g$ is also injective, $g(f(x)) \ne g(f(x'))$. Therefore, $(g \circ f)(x) \ne (g \circ f)(x')$.
	
	\subsubsection*{Proving that $f$ and $g$ being surjective implies that $g \circ f$ is too}
	
	We will show that $g \circ f$ is surjective by showing that for any $c \in Z$, there exists some $a \in X$ such that $(g \circ f)(a) = c$. Since $g$ is surjective, there exists some $b \in Y$ such that $g(b) = c$. Because $f$ is also surjective, there exists some $a \in X$ such that $f(a) = b$. Therefore, $(g \circ f)(a) = c$, and $g \circ f$ is surjective.
	
	\subsection*{Exercise 3.3.3}
	
	\subsubsection*{Finding when the empty function is injective}
	
	We will prove that the empty function $f \colon \emptyset \rightarrow X$ is injective for any $X$. Since we cannot find any $x, x' \in \emptyset$ that are unequal, the empty function is (vacuously) injective.
	
	\subsubsection*{Finding when the empty function is surjective}
	
	We will prove that the empty function $f \colon \emptyset \rightarrow X$ is surjective only when $X = \emptyset$. The statement
	\begin{center}
		``For every $y \in X$, there exists $x \in \emptyset$ such that $f(x) = y$''
	\end{center}
	can only be true when it is impossible that $y \in X$, because $x \in \emptyset$ is impossible. But then we have that $X = \emptyset$, and we have proved our claim.
	
	\subsubsection*{Finding when the empty function is bijective}
	
	We will prove that the empty function $f \colon \emptyset \rightarrow X$ is bijective only when $X = \emptyset$. Since $f$ is always injective regardless of the choice of $X$, the empty function being bijective is equivalent to it being surjective. Therefore, since $f$ is surjective precisely when $X = \emptyset$, the empty function is bijective only when $X = \emptyset$.
	
	\subsection*{Exercise 3.3.4}
	
	\subsubsection*{Showing that if $g \circ f = g \circ \tilde{f}$ and $g$ is injective, then $f = \tilde{f}$}
	
	We will use proof by contradiction. If $f \ne \tilde{f}$, then for some $x \in X$, we have $f(x) \ne \tilde{f}(x)$. Therefore, since $g$ is injective, $(g \circ f)(x) \ne (g \circ \tilde{f})(x)$, and $g \circ f \ne g \circ \tilde{f}$. This is a contradiction, and therefore, $f = \tilde{f}$. This statement is not necessarily true if $g$ is not injective. If $X = Y = Z = \mathbf{N}$, $f(x) = 0$, $\tilde{f}(x) = 1$, and $g(x) = 0$, then $f \ne \tilde{f}$, while $(g \circ f)(x) = 0 = (g \circ \tilde{f})(x)$.
	
	\subsubsection*{Showing that if $g \circ f = \tilde{g} \circ f$ and $f$ is surjective, then $g = \tilde{g}$}
	
	We will use proof by contradiction again. If $g \ne \tilde{g}$, then for some $y \in Y$, we have $g(y) \ne \tilde{g}(y)$. Since $f$ is surjective, there exists $x \in X$ such that $f(x) = y$. But then $(g \circ f)(x) \ne (\tilde{g} \circ f)(x)$. This is a contradiction, and therefore, $g = \tilde{g}$. This statement is not necessarily true if $f$ is not surjective. If $X = Y = Z = \mathbf{N}$, $f(x) = 0$, $g(x) = x$, and $\tilde{g}(x) = 0$, then $g \ne \tilde{g}$, even though $(g \circ f)(x) = 0 = (\tilde{g} \circ f)(x)$.
	
	\subsection*{Exercise 3.3.6}
	
	If $f(x) = a$, then by the definition of $f^{-1}$, $f^{-1}(f(x)) = f^{-1}(a) = x$. If $f^{-1}(y) = b$, then $f(b) = y$, so $f(f^{-1}(y)) = f(b) = y$.
	
	We can deduce that $f^{-1}$ is bijective from Exercise 3.3.5 and the fact that the identity map $\imath_{X \rightarrow X}$ defined as $\imath_{X \rightarrow X}(x) = x$ for all $x \in X$ is obviously bijective.
	
	\subsection*{Exercise 3.3.7}
	
	By Exercise 3.3.2, $g \circ f$ is both injective and surjective, and is therefore bijective. The only thing left we have to prove is that $(f^{-1} \circ g^{-1} \circ g \circ f)(x) = x$. But $(f^{-1} \circ g^{-1} \circ g \circ f)(x) = (f^{-1} \circ f)(x) = x$, and we have proved what we want.
	
	\section*{Section 3.4}
	
	\subsection*{Challenge to define $f(S)$ using the axiom of specification}
	
	We can define $f(S) \coloneqq \{y \in Y: \text{there exists } x \in S \text{ such that } f(x) = y\}$.
	
	\subsection*{Exercise 3.4.1}
	
	First, we will show that any element of the forward image of $V$ under $f^{-1}$ is contained in the inverse image of $V$ under $f$. If $x$ is in the forward image of $V$ under $f^{-1}$, then there exists $v$ such that $f^{-1}(v) = x$ and therefore $f(x) = v$. But then $x$ is in the inverse image of $V$ under $f$.
	
	Next, we will show that any element $v$ of the inverse image of $V$ under $f$ is contained in the forward image of $V$ under $f^{-1}$. Since there exists $x$ such that $f(x) = v$ and therefore $f^{-1}(v) = x$, $x$ is in the forward image of $V$ under $f$.
	
	Finally, we have that the forward image of $V$ under $f^{-1}$ is equal to the inverse image of $V$ under $f$, and that it is valid to use the notation $f^{-1}(V)$.
	
	\subsection*{Exercise 3.4.6}
	
	By Axiom 3.11, $\{0, 1\}^X$ exists. We can use Axiom 3.7 (the axiom of replacement) to create a set \[Y = \{f^{-1}(\{1\}): f \in \{0, 1\}^X\}.\] Every $S$ which is a subset of $X$ is in $Y$, because defining $f(x)$ to be $1$ when $x$ is in $S$ and to be $0$ otherwise means that $f^{-1}(\{1\}) = S$. Every element of $Y$ is a subset of $X$, because every element of $f^{-1}(\{1\})$ has to be in $X$.
	
	\subsection*{Exercise 3.4.9}
	
	We will show that \[\{x \in A_\beta: x \in A_\alpha \text{ for all } \alpha \in I\} = \{x \in A_{\beta'}: x \in A_\alpha \text{ for all } \alpha \in I\}.\] We know that $y \in \{x \in A_\beta: x \in A_\alpha \text{ for all } \alpha \in I\}$ if and only if $y \in A_\beta$ and $y \in A_\alpha$ for all $\alpha \in I$. But the latter statement implies the former, so $y \in \{x \in A_\beta: x \in A_\alpha \text{ for all } \alpha \in I\}$ if and only if $y \in A_\alpha$ for all $\alpha \in I$.
	
	We can do something similar for $A_{\beta'}$. Therefore, \[\{x \in A_\beta: x \in A_\alpha \text{ for all } \alpha \in I\} = \{x \in A_{\beta'}: x \in A_\alpha \text{ for all } \alpha \in I\}.\]
	
	\section*{Section 3.5}
	
	\subsection*{Exercise 3.5.1}
	
	First, we will show that $(x, y) = (x', y')$ implies that both $x = x'$ and $y = y'$. Since $(x, y) \coloneqq \{\{x\}, \{x, y\}\}$, we have to show that $\{\{x\}, \{x, y\}\} = \{\{x'\}, \{x', y'\}\}$ implies that both $x = x'$ and $y = y'$. If $\{\{x\}, \{x, y\}\} = \{\{x'\}, \{x', y'\}\}$, then $\{x\} \in \{\{x'\}, \{x', y'\}\}$. In order for this to be true, we must have $\{x\} = \{x'\}$, which implies $x = x'$, or $\{x\} = \{x', y'\}$, which implies $x' = y' = x$. But the latter case implies the former case, so we can ignore the latter one. Doing the same with $\{x, y\}$, $\{x'\}$, and $\{x', y'\}$ yields that $x = x'$ and $y = y'$. The converse is true by Axiom 3.2.
	
	We can show that $X \times Y$ is a set if $X$ and $Y$ are both sets using Axiom 3.1 (sets are objects), Axiom 3.7 (axiom of replacement), and Axiom 3.12 (axiom of union). By Axioms 3.1 and 3.7, we can create the set $Z = \{\{(x, y)\}: y \in Y\}: x \in X\}$. By Axiom 3.12, we can also create the set $\bigcup Z$, which we can define to be $X \times Y$.
	
	I won't do the additional challenge.
	
	\subsection*{Exercise 3.5.6}
	
	We will show that if $A \times B \subseteq C \times D$ and $A, B, C, D \ne \emptyset$, then $A \subseteq C$ and $B \subseteq D$. Since $A$ and $B$ are nonempty, by the single choice lemma (Lemma 3.1.5), there exist some $a \in A$ and $b \in B$. Now, for any $x \in A$, since $(x, b) \in A \times B \subseteq C \times D$, we have $x \in C$. Similarly, for any $y \in B$, since $(a, y) \in A \times B \subseteq C \times D$, we have $y \in D$. Therefore, $A \subseteq C$ and $B \subseteq D$.
	
	However, we can show that if we remove the assumption that $A$, $B$, $C$, and $D$ are nonempty, then this statement is false. If $A = \{0\}$, $C = \{1\}$, and $B = D = \emptyset$, then $A \times B$ is empty, as there are no elements of $B$. Similarly, $C \times D = \emptyset$, and therefore $A \times B \subseteq C \times D$. However, $A \not \subseteq C$.
	
	If $A \subseteq C$ and $B \subseteq D$, all ordered pairs $(a, b)$ where $a \in A$ and $b \in B$ are in $C \times D$ because $a \in C$ and $b \in D$. Hence, $A \times B \subseteq C \times D$. This statement holds even when $A$, $B$, $C$, and $D$ are not all nonempty.
	
	Assuming that $A \times B = C \times D$ and $A, B, C, D \ne \emptyset$, $A = C$ and $B = D$. This is because $A \times B \subseteq C \times D$ and $C \times D \supseteq A \times B$, and therefore $A \subseteq C$, $A \supseteq C$, $B \subseteq D$, and $B \supseteq D$.
	
	If we remove the assumption that $A, B, C, D \ne \emptyset$, the statement is false. The counterexample from before ($A = \{0\}$, $C = \{1\}$, $B = D = \emptyset$) also works here. We have that $A \times B = C \times D = \emptyset$, but $A \ne C$.
	
	We have that $A = C$ and $B = D$ imply that $A \times B = C \times D$ because $A \subseteq C$ and $B \subseteq D$ imply, together, that $A \times B \subseteq C \times D$ and the same with $\subseteq$ replaced by $\supseteq$.
	
	\subsection*{Exercise 3.5.13 (not done yet)}
	
	
	
	\section*{Section 3.6}
	
	\subsection*{Exercise 3.6.2}
	
	If $X$ has cardinality $0$, there exists a bijection $f \colon X \rightarrow \{i \in \mathbf{N}: 1 \le i \le 0\}$. The set $\{i \in \mathbf{N}: 1 \le i \le 0\}$ is obviously empty. If $X \ne \emptyset$, then there exists some $x \in X$. But then $f(x)$ is in $\emptyset$, and therefore $X$ must be empty.
	
	If $X$ is empty, the empty function $f \colon \emptyset \rightarrow \emptyset$ is a bijection (injectivity is because there are no two elements in the domain at all, and surjectivity is from that the range is empty)
	
\end{document}